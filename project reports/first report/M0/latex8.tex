
%
%  $Description: Author guidelines and sample document in LaTeX 2.09$ 
%
%  $Author: Ting-Wu Chin and Ahmet Fatih Inci $
%  $Date: 2017/9/29 $
%  $Revision: 1.?? $
%

\documentclass[times, 10pt,twocolumn]{article} 
\usepackage{latex8}
\usepackage{times}

%\documentstyle[times,art10,twocolumn,latex8]{article}

%------------------------------------------------------------------------- 
% take the % away on next line to produce the final camera-ready version 
\pagestyle{empty}

%------------------------------------------------------------------------- 

\begin{document}

\title{Power/Performance Analysis and Optimization for Deep Learning on CPU-GPU Platforms}

\author{Ahmet Fatih Inci and Ting-Wu Chin  \\
Carnegie Mellon University\\ Department of Electrical and Computer Engineering \\ \{ainci, tingwuc\}@andrew.cmu.edu\\
}
\maketitle
\thispagestyle{empty}

\begin{abstract}
    Due to the intrinsic data parallelism characteristic of deep learning, GPU is much a better platform for deep learning applications compared to CPU. This works aim at understanding what is left to be done for CPU given that GPU has to run deep learning applications on embedded platform. In modern mobile SoC, e.g. Nvidia Tegra X1, GPU and CPU shares same power budget and memory budget, and hence affects each other cohesively. By characterizing different deep learning workloads and CPU workloads under different kind of CPU-GPU frequencies, we wish to understand what is left to be done for CPU and how those workloads affect the tasks on GPU. 

\end{abstract}

\Section{Introduction}
\input{"introduction.tex"}
\Section{Related Work}
\input{"relatedwork.tex"}
\Section{Methodology}
\input{"methodology"}
\Section{Objectives and Deliverables}
\input{"objectives.tex"}
\Section{Timeline}
Timeline of the project tasks are listed below. Group members worked closely on each task to come up with a thorough analysis of power and performance results for deep learning on a CPU-GPU platform. Team members worked jointly on this report and analysis.  

\begin{itemize}
\item[M2 -] Running CPU-GPU benchmarks jointly by changing frequency values for both CPU and GPU
\item[-] Analyzing power, performance, and temperature results 
\end{itemize}
\Section{Conclusion}
In this project, we analyze power and performance results of various deep neural networks running jointly with various CPU benchmarks on a CPU-GPU platform which is NVIDIA Jetson TX1. Power and performance analysis identify what is left to be done on CPU while GPU is running inference on deep neural network. 

Our comprehensive analysis (81 results with different configurations) shows us the best CPU benchmarks to run with various DNN configurations. Our results show that although people focus on GPU for running DNNs, role of CPU is not negligible in deep learning applications. CPU frequency still plays a significant role due to data preparation stage. Another important conclusion is that memory consumption of both CPU and GPU workloads affects overall power and performance. Therefore, memory consumption of both CPU and GPU benchmarks should be taken into consideration while running CPU-GPU benchmarks together. 

Group members worked closely on each task to come up with a thorough analysis of power and performance results for deep learning on a CPU-GPU platform. Team members worked jointly on this report and analysis.  

Project materials are available in the project website. \textit{https://github.com/afinci/18-743-Power-and-Performance-optimizations-for-DNNs-on-CPU-GPU}
% If there are cited but unreferenced resources, place them in the brackets 
% on the line below.
% \nocite{}
\bibliographystyle{latex8}
\bibliography{latex8}

\end{document}

